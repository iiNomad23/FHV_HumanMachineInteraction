\documentclass[sigconf,natbib=false,10pt]{acmart}

%%
%% \BibTeX command to typeset BibTeX logo in the docs
\AtBeginDocument{%
	\providecommand\BibTeX{{%
			Bib\TeX}}}

%% Rights management information.  This information is sent to you
%% when you complete the rights form.  These commands have SAMPLE
%% values in them; it is your responsibility as an author to replace
%% the commands and values with those provided to you when you
%% complete the rights form.
%\setcopyright{acmlicensed}
%\copyrightyear{2018}
%\acmYear{2018}
%\acmDOI{XXXXXXX.XXXXXXX}

%% Bibliography style
\RequirePackage[datamodel=acmdatamodel,style=acmnumeric]{biblatex}

%% Declare bibliography sources
\addbibresource{reference.bib}

%%
%% end of the preamble, start of the body of the document source.
\begin{document}
	
	%%
	%% The "title" command has an optional parameter,
	%% allowing the author to define a "short title" to be used in page headers.
	\title{Assistant Tools and Accessibility Features for Blind People Playing Visual-Centric Digital Games}
	
	\author{Marco Prescher}
	%\authornote{This is a author note!}
	\affiliation{%
		\institution{FHV University of Applied Sciences}
		\streetaddress{Hochschulstraße 1}
		\city{Dornbirn}
		\state{Vorarlberg}
		\country{Austria}}
		\postcode{6850}
	\email{marco.prescher@students.fhv.at}
	
	%%
	%% By default, the full list of authors will be used in the page
	%% headers. Often, this list is too long, and will overlap
	%% other information printed in the page headers. This command allows
	%% the author to define a more concise list
	%% of authors' names for this purpose.
	\renewcommand{\shortauthors}{Marco Prescher}
	
	%%
	%% The abstract is a short summary of the work to be presented in the
	%% article.
	%TODO
	\begin{abstract}
		Lorem ipsum dolor sit amet, consectetur adipiscing elit. Morbi
		malesuada, quam in pulvinar varius, metus nunc fermentum urna, id
		sollicitudin purus odio sit amet enim. Aliquam ullamcorper eu ipsum
		vel mollis. Curabitur quis dictum nisl. Phasellus vel semper risus, et
		lacinia dolor. Integer ultricies commodo sem nec semper.
	\end{abstract}
	
	\ccsdesc[500]{Applied computing~Computer games}
	\ccsdesc[500]{Human-centered computing~Accessibility}
	\ccsdesc[500]{Human computer interaction (HCI)}
	
	%%
	%% Keywords. The author(s) should pick words that accurately describe
	%% the work being presented. Separate the keywords with commas.
	\keywords{blind, accessibility, gaming, digital games, navigation, tools, AI}
	
	%TODO?
	%\received{20 February 2007}
	%\received[revised]{12 March 2009}
	%\received[accepted]{5 June 2009}
	
	%%
	%% This command processes the author and affiliation and title
	%% information and builds the first part of the formatted document.
	\maketitle
	
	\section{Introduction}
	Today's accessible games for blind people are mainly games which are directly developed for them (\textcite{goncalves_my_2023}).
	While these games are enjoyable, mainstream games are a serious challenge for blind people because they consist of complex environments, mechanics and interactions with \emph{Non-Player Character} (NPC) players or even real players in \emph{Player versus player} (PvP) games.
	
	One big step forward making mainstream games more accessible for blind people was the game \emph{The Last of Us Part II} (TLOU2) \cite{playstation_last_2020, playstation_last_2020-1}. 
	According to \textcite{leite_extended_2021} the game company \emph{Naughty Dog} implemented more than 60 accessibility features and is considered as the most accessible game ever produced.
	Additionally, \textcite{dale_last_2024} described that the game can be played all the way through with audio cues and navigation aids.
	It includes preset accessibility options for common disabilities like hearing or vision impairments. 
	It also introduces accessibility menus when the game is first started, making it easier for players with disabilities to adjust settings.
	To top that, \emph{Naughty Dog} released a remastered version of TLOU2 in 2024 with a reworked \emph{Cinematic Audio Descriptions} feature \cite{playstation_last_2024}.
	
	Implementing accessibility features is much needed in digital games to ensure that everyone can enjoy gaming.
	However, game developers in general face various problems in the process of developing games, some of them are:
	
	\begin{itemize}
		\setlength\itemsep{0.5em}
		\item Diverse Needs
		\item Technical Challenges
		\item Design Compromises
		\item Standardization
	\end{itemize}
	
	In this study, we explore different and innovative accessibility features and tools, including haptic feedback which is a major technical challenge, design compromises and their associated problems and the need for standardization.
	This raises two relevant research questions (RQ):
	
	\begin{itemize}
		\setlength\itemsep{0.5em}
		\item RQ1: Which innovative accessibility features and tools can enhance the gaming experience for blind players?
		\item RQ2: In what ways can haptic feedback enhancements improve the game experience for blind players?
	\end{itemize}
	
	\section{The Problem}
	Blind players encounter many different barriers when playing visual-centric digital games which often rely greatly on graphical interfaces and visual cues. 
	To top that, the collection of those mainstream games have different perspectives such as top-down, first-person, and third-person views, where all three views give the player unique challenges in navigating game environment, understanding game objectives and interacting with in-game elements like players or objects.
	Building on that, the authors of \textcite{goncalves_my_2023} have categorized seven themes and identified unresolved barriers (see \autoref{fig:seven-themes}) which still represent a great challenge for both players and developers.
	
	\begin{figure*}[ht]
		\centering
		\includegraphics[scale=0.6,width=\textwidth]{assets/seven-themes.png}
		\caption{Seven themes and respective unresolved barriers (Source: \textcite{goncalves_my_2023})}
		\label{fig:seven-themes}
	\end{figure*}

	\autoref{fig:seven-themes} gives a great overview what accessibility features and assistant tools are still missing and in which direction the gaming industry should focus.
	
	\section{My Idea}
	The gaming industry came a long way from no accessibility features and assistant tools at all to implementing more than 60 accessibility features in one game \cite{playstation_last_2020}.
	According to research papers \cite{goncalves_my_2023, grammenos_designing_2009, grammenos_game_2008, araujo_mobile_2017} some of the most important accessibility features for blind people in visual-centric games include:
	
	\begin{itemize}
		\setlength\itemsep{0.5em}
		\item Audio Cues and Descriptions
		\item \emph{Text-to-Speech} (TTS) and Voiceover
		\item Navigation Aids and Wayfinding Tools
		\item Comprehensive Audio Design
		\item Customizable Controls and Inputs
		\item Tactile Feedback and Controller Design
	\end{itemize}
    
    Some of these have already been implemented to a certain extent in some games, but as \autoref{fig:seven-themes} notes, there are still problems.
    Especially when it comes to the environment, pathfinding, perspective and interacting with the world, blind players face major challenges, which according to \textcite{goncalves_my_2023} could mean they stop playing these games because they simply can not find the right way to play.
    
    To address the environment and pathfinding problems, one new technology was introduced in 2018 by \textcite{andrade_echo-house_2018} to use echolocation to explore a virtual environment which could drastically improve navigation in it.
    As for perspective (camera) and interacting with the world, hardware solutions like haptic feedback or AI assistant tools could be a solution when developed and integration further.
    Whereas the haptic feedback \cite{bello_haptics_2016} of for example PS5-Controllers \cite{akyaman_anticipated_2021, chen_gamepad_2024} could indicate when players aim too high or too low while the AI-Tool could provide the player with enhanced audio descriptions and how to interact with the world.
    
    In summary, it can be said that the game industry already takes into account the most important accessibility features, yet most modern visually-centric digital games are still a major challenge for blind players.
    In the following section we will delve deeper into some of the listed features above, their problems and possible solutions, as well as how to improve the overall experience of blind people.
	
	\section{The Details}
	In this section, we delve into enhancing accessibility in digital games for blind players.
	Firstly, we aim to explore previously mentioned developing problems, from universally accessible game design affecting diverse needs followed by technical challenges.
	Secondly, we propose enhancements of already existing accessibility features in the theme environment navigating.
	Lastly, we explore various types of feedback which are essential in enhancing of accessibility in digital games and focus on haptic feedback as well as further possibilities to use that to improve the experience in digital games for blind players.
	
	\subsection{Universally accessible game design}
	%TODO
	Test
	
	
	\subsection{Navigate a virtual environment using echolocation}
	%TODO
	Test
	
	\subsection{Assistive haptic feedback}
	%TODO
	Short introduction into feedback variants and where it is used "visual perception (sight), haptics (tactile) and aural (hearing)". \cite{kuber_towards_2007}
	
	\subsection{Further possibilities to use haptic feedback}
	%TODO
	Test
	
	\section{Related Work}
	Accessibility in visual-centric games is getting more important and the game industry have to keep up implementing them.
	The research paper \textcite{goncalves_my_2023} has shown us that there are still major challenges to improve and further enhance existing accessibility features in various themes.
	In 2008 \textcite{grammenos_game_2008} developed a game called \emph{Game Over!}, which is the first universally inaccessible game showing people how it is if a game is unplayable.
	Additionally, in 2018 \textcite{andrade_echo-house_2018} described a new method to explore a virtual environment by using echolocation.
	This section provides a deeper insight into these studies, how they approached the topic and what results they achieved.
	
	\subsection{Echolocation in games}
	Test
	
	\section{Conclusions and Further Work}
	%TODO
	A recap of the problem addressed and the proposed solution.
	The implications of your research for the gaming industry and the broader accessibility community.
	Any remaining challenges or unanswered questions that need to be addressed.
	Suggestions for future research directions or enhancements to your proposed idea.
	How your work contributes to advancing the state-of-the-art in accessibility technology for blind individuals playing digital games.
	
	
	%%
	%% Print the bibliography
	%%
	\printbibliography
	
\end{document}