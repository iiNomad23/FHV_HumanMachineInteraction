\documentclass[sigconf,natbib=false]{acmart}

%%
%% \BibTeX command to typeset BibTeX logo in the docs
\AtBeginDocument{%
	\providecommand\BibTeX{{%
			Bib\TeX}}}

%% Rights management information.  This information is sent to you
%% when you complete the rights form.  These commands have SAMPLE
%% values in them; it is your responsibility as an author to replace
%% the commands and values with those provided to you when you
%% complete the rights form.
%\setcopyright{acmlicensed}
%\copyrightyear{2018}
%\acmYear{2018}
%\acmDOI{XXXXXXX.XXXXXXX}

%% Bibliography style
\RequirePackage[datamodel=acmdatamodel,style=acmnumeric]{biblatex}

%% Declare bibliography sources
\addbibresource{reference.bib}

%%
%% end of the preamble, start of the body of the document source.
\begin{document}
	
	%%
	%% The "title" command has an optional parameter,
	%% allowing the author to define a "short title" to be used in page headers.
	\title{Assistant tools for blind people playing games}
	
	\author{Marco Prescher}
	%\authornote{This is a author note!}
	\affiliation{%
		\institution{FHV University of Applied Sciences}
		\streetaddress{Hochschulstraße 1}
		\city{Dornbirn}
		\state{Vorarlberg}
		\country{Austria}}
		\postcode{6850}
	\email{marco.prescher@students.fhv.at}
	
	%%
	%% By default, the full list of authors will be used in the page
	%% headers. Often, this list is too long, and will overlap
	%% other information printed in the page headers. This command allows
	%% the author to define a more concise list
	%% of authors' names for this purpose.
	\renewcommand{\shortauthors}{Trovato et al.}
	
	%%
	%% The abstract is a short summary of the work to be presented in the
	%% article.
	%TODO
	\begin{abstract}
		Lorem ipsum dolor sit amet, consectetur adipiscing elit. Morbi
		malesuada, quam in pulvinar varius, metus nunc fermentum urna, id
		sollicitudin purus odio sit amet enim. Aliquam ullamcorper eu ipsum
		vel mollis. Curabitur quis dictum nisl. Phasellus vel semper risus, et
		lacinia dolor. Integer ultricies commodo sem nec semper.
	\end{abstract}
	
	\ccsdesc[500]{Applied computing~Computer games}
	\ccsdesc[500]{Human-centered computing~Accessibility}
	\ccsdesc[500]{Human computer interaction (HCI)}
	
	%%
	%% Keywords. The author(s) should pick words that accurately describe
	%% the work being presented. Separate the keywords with commas.
	\keywords{blind, accessibility, gaming, digital games, navigation, tools, AI}
	
	%TODO?
	%\received{20 February 2007}
	%\received[revised]{12 March 2009}
	%\received[accepted]{5 June 2009}
	
	%%
	%% This command processes the author and affiliation and title
	%% information and builds the first part of the formatted document.
	\maketitle
	
	\section{Introduction}
	%TODO
	Short summary of zelda cane article and what i want to expore more!
	
	\section{Related Work}
	%TODO
	Maybe short summary of the related works i gatherd (State of the art)

	\section{Conclusion}
	%TODO
	\cite{westland_cost_2002}
	
	
	%%
	%% Print the bibliography
	%%
	\printbibliography
	
\end{document}