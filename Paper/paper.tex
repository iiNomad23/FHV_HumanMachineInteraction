\documentclass[sigconf,natbib=false,10pt]{acmart}

%%
%% \BibTeX command to typeset BibTeX logo in the docs
\AtBeginDocument{%
	\providecommand\BibTeX{{%
			Bib\TeX}}}

%% Rights management information.  This information is sent to you
%% when you complete the rights form.  These commands have SAMPLE
%% values in them; it is your responsibility as an author to replace
%% the commands and values with those provided to you when you
%% complete the rights form.
%\setcopyright{acmlicensed}
%\copyrightyear{2018}
%\acmYear{2018}
%\acmDOI{XXXXXXX.XXXXXXX}

%% Bibliography style
\RequirePackage[datamodel=acmdatamodel,style=acmnumeric]{biblatex}

%% Declare bibliography sources
\addbibresource{reference.bib}

%%
%% end of the preamble, start of the body of the document source.
\begin{document}
	
	%%
	%% The "title" command has an optional parameter,
	%% allowing the author to define a "short title" to be used in page headers.
	\title{Assistant Tools and Accessibility Features for Blind People Playing Visual-Centric Digital Games}
	
	\author{Marco Prescher}
	%\authornote{This is a author note!}
	\affiliation{%
		\institution{FHV University of Applied Sciences}
		\streetaddress{Hochschulstraße 1}
		\city{Dornbirn}
		\state{Vorarlberg}
		\country{Austria}}
		\postcode{6850}
	\email{marco.prescher@students.fhv.at}
	
	%%
	%% By default, the full list of authors will be used in the page
	%% headers. Often, this list is too long, and will overlap
	%% other information printed in the page headers. This command allows
	%% the author to define a more concise list
	%% of authors' names for this purpose.
	\renewcommand{\shortauthors}{Trovato et al.}
	
	%%
	%% The abstract is a short summary of the work to be presented in the
	%% article.
	%TODO
	\begin{abstract}
		Lorem ipsum dolor sit amet, consectetur adipiscing elit. Morbi
		malesuada, quam in pulvinar varius, metus nunc fermentum urna, id
		sollicitudin purus odio sit amet enim. Aliquam ullamcorper eu ipsum
		vel mollis. Curabitur quis dictum nisl. Phasellus vel semper risus, et
		lacinia dolor. Integer ultricies commodo sem nec semper.
	\end{abstract}
	
	\ccsdesc[500]{Applied computing~Computer games}
	\ccsdesc[500]{Human-centered computing~Accessibility}
	\ccsdesc[500]{Human computer interaction (HCI)}
	
	%%
	%% Keywords. The author(s) should pick words that accurately describe
	%% the work being presented. Separate the keywords with commas.
	\keywords{blind, accessibility, gaming, digital games, navigation, tools, AI}
	
	%TODO?
	%\received{20 February 2007}
	%\received[revised]{12 March 2009}
	%\received[accepted]{5 June 2009}
	
	%%
	%% This command processes the author and affiliation and title
	%% information and builds the first part of the formatted document.
	\maketitle
	
	\section{Introduction}
	Today's accessible games for blind people are mainly games which are directly developed for them (\textcite{goncalves_my_2023}).
	While these games are enjoyable, mainstream games are a serious challenge for blind people because they consist of complex environments, mechanics and interactions with \emph{Non-Player Character} (NPC) players or even real players in \emph{Player versus player} (PvP) games.
	
	One big step forward making mainstream games more accessible for blind people was the game \emph{The Last of Us Part II} (TLOU2) \cite{playstation_last_nodate, playstation_last_nodate-1}. 
	According to \textcite{leite_extended_2021} the game company \emph{Naughty Dog} implemented more than 60 accessibility features and is considered as the most accessible game ever produced.
	Additionally, \textcite{dale_last_2024} described that the game can be played all the way through with audio cues and navigation aids.
	It includes preset accessibility options for common disabilities like hearing or vision impairments. 
	It also introduces accessibility menus when the game is first started, making it easier for players with disabilities to adjust settings.
	To top that, \emph{Naughty Dog} released a remastered version of TLOU2 in 2024 with a reworked \emph{Cinematic Audio Descriptions} feature \cite{playstation_last_nodate-2}.
	
	In this work we go deeper into different and potentially new accessibility features as well as what assistant tools blind players use and how software companies implement them.
	This raises two relevant research questions (RQ):
	
	\begin{itemize}
		\item RQ1: What new breakthrough accessibility features and tools could enhance the gaming experience for blind players?
		\item RQ2: How can the development and implementation of these features and tools be standardized within the development cycle of visual-centric digital games?
	\end{itemize}
	
	\section{The Problem}
	%TODO
	Summary of findings in "MyZeldaCane" and what current problems with visual centric games are (for blind people).
	Lack of new tools and features as well as costs for standardized the development cycle to implement those.
	
	\section{My Idea}
	%TODO
	Discuss your proposed idea for addressing the problem identified in the previous section. This could involve developing new assistant tools or enhancing existing accessibility features for blind people playing visual-centric digital games. Outline the key components of your idea and how it aims to improve the gaming experience for blind players. Consider addressing the following points:
	
	The rationale behind your idea and why it is important.
	How your idea builds upon existing tools and features.
	Any innovative approaches or technologies you plan to implement.
	The potential impact of your idea on blind gamers and the gaming industry as a whole.
	
	\section{The Details}
	%TODO
	New tools for blind players and newest Accessibility Features software companies using/developing.
	
	Screen Reader
	Audio Cues and Descriptions
	Haptic Feedback
	Customizable Controls
	Text-to-Speech and Speech Recognition
	Accessible Menus and Interfaces
	Echolocation
	
	Provide a detailed explanation of how your idea will be implemented. This section should delve into the technical aspects of your proposed solution and address any challenges or limitations that may arise. Consider including the following information:
	
	The specific features and functionalities of your assistant tools or accessibility features.
	The technologies or algorithms involved in implementing these features (e.g., machine learning, computer vision).
	Any hardware or software requirements necessary for your solution to work.
	How your solution will be integrated into existing digital games or gaming platforms.
	Any user testing or feedback you have conducted to validate your idea.
	
	\section{Related Work}
	%TODO
	Review existing literature and research related to assistant tools and accessibility features for blind gamers. This section should provide an overview of the current state-of-the-art in this field and highlight any relevant studies or projects. Consider addressing the following points:
	
	Previous research on accessibility features in digital games for blind players.
	Existing assistant tools or software developed to improve the gaming experience for blind individuals.
	Studies or projects that have investigated the challenges faced by blind gamers and potential solutions.
	Any notable advancements or innovations in the field of accessibility technology for blind users.
	How your research builds upon or contributes to the existing body of work in this area.
	
	\section{Conclusions and Further Work}
	What else could be done, explored deeper or would benefit blind players
	%TODO
	Summarize the key findings and contributions of your research paper. Reflect on the significance of your proposed idea and its potential impact on improving accessibility for blind gamers. Additionally, discuss any avenues for future research or development in this field. Consider addressing the following points:
	
	A recap of the problem addressed and the proposed solution.
	The implications of your research for the gaming industry and the broader accessibility community.
	Any remaining challenges or unanswered questions that need to be addressed.
	Suggestions for future research directions or enhancements to your proposed idea.
	How your work contributes to advancing the state-of-the-art in accessibility technology for blind individuals playing digital games.
	
	
	%%
	%% Print the bibliography
	%%
	\printbibliography
	
\end{document}